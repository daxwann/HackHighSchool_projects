%******************************************************************************%
%                                                                              %
%                  sample.en.tex for LaTeX                                     %
%                  Created on : Tue Mar 10 13:27:28 2015                       %
%                  Made by : David "Thor" GIRON <thor@42.fr>                   %
%                                                                              %
%******************************************************************************%

\documentclass{42-en}


%******************************************************************************%
%                                                                              %
%                                    Header                                    %
%                                                                              %
%******************************************************************************%
\begin{document}



                           \title{Sample document}
                          \subtitle{Awesome subtitle}
                       \member{Your Name}{your@mail.com}
                        \member{42 Staff}{pedago@42.fr}

\summary {
  This document is a sample and an introduction to \texttt{LaTeX} and
  the homebrew style class from \href{www.42.fr}{42}.
}

\maketitle

\tableofcontents


%******************************************************************************%
%                                                                              %
%                                  Foreword                                    %
%                                                                              %
%******************************************************************************%
\chapter{Foreword}

    The forewords section of a \texttt{42} subject is usually not
    related in any way to the actual topic of the subject. The idea is
    to share some jokes (often questionable) or something that the
    community might be interested in.\\





    % Spacing in the source code does not influence spacing in the
    % generated pdf. The blank lines aboves and below won't appear.
    % Instead, use \newline (or its shortcut \\) and \newpage to
    % create vertical spacing.





    As a consequence, let's use the forewords section of this sample
    \texttt{42} subject to introduce the contents of this document and
    its goals. In particular, the formating of a trivial
    \texttt{LaTeX} document and the normalized chaptering of our
    subjects. If you read this from the pdf, don't forget to open the
    source file (file \texttt{sample.en.tex}) next to this pdf, in
    order to see behind the scenes, and to understand which command
    generates which result. Otherwise, if you have started with the
    sources, congrats, that's the spirit ! But open the pdf (file
    \texttt{sample.en.pdf}) anyway.\\

    What to do if the file \texttt{sample.en.pdf} is not available ?
    Easy, just compile the source file \texttt{sample.en.tex} using
    the shell command \texttt{make}. Please refer to the documentation
    to set up \texttt{LaTeX} on your system if needed.\\

    If you're not familliar with \texttt{LaTeX}'s syntax, here is a
    fairly exhaustive list of everything you'll need to write your
    subject.\\


    \section{Example of section}


        \subsection{Example of sub-section}

           This sub-section is empty.


        \newpage


        \subsection{A bullet point list}

            \begin{itemize}\itemsep1pt
                \item what
                \item a
                \item wonderful
                \item list.\\
            \end{itemize}


        \subsection{A descriptions list}

            \begin{description}\itemsep3pt
                \item [Orange:] Round and orange fruit.
                \item [Strawberry:] Strawberry shaped fruit. Also red.
                \item [Cucumber:] Phallus shaped and green vegetable.\\
            \end{description}


        \subsection{An enumeration}

            An enumeration of the reasons why I like you:\\

            \begin{enumerate}\itemsep7pt
                \item You are smart.
                \item Your are very talented.
                \item Your are magnificent.
                \item I'm a nice person.
            \end{enumerate}


        \subsection{Urls and links}

            If you have no clue how to insert links or urls in your
            document, search for an online explanation using
            \href{www.google.com}{Google}. Please note that Google is
            available at \url{www.google.com}.


        \newpage


        \subsection{An info box}

            \info{
              For information, please read this info box.
            }


        \subsection{A hint box}

            \hint {
              You should read this hint box, really.
            }


        \subsection{A warning box}

            \warn {
              Beware ! This is a warning box !
            }


        \newpage


        \subsection{A \texttt{shell} snippet}

           \begin{42console}
sudo rm -rf /\end{42console}



        \subsection{A \texttt{C} code snippet}

           \begin{42ccode}
int main( void ) {

    puts( "hello world !" );
    return 0;
}
\end{42ccode}


        \subsection{A \texttt{C++} code snippet}

            \begin{42cppcode}
int main( void ) {

    std::cout << "hello world !" << std::endl;
    return 0;
}
\end{42cppcode}


        \subsection{A \texttt{Python} code snippet}

           \begin{minted}{python}
           def function(var):
            print(var)
           \end{minted}
           


        \subsection{A picture !}

            \begin{figure}[H]
                \begin{center}
                    \includegraphics[width=8cm]{42.png}
                \end{center}
            \end{figure}


        \newpage


        \subsection{Some special characters}

            \begin{description}\itemsep1pt
                \item [Underscore :] \_
                \item [Ampersand :] \&
                \item [Dollar :] \$
                \item [Elipsis :] \dots
            \end{description}


    \section{About chaptering}

    Each chapter of the pdf must be present in your subject,
    \textbf{including} the \texttt{Forewords} chapter. For your
    confort, the best way to use this sample \texttt{LaTeX} file is to
    copy it and rename it, then replace the provided descriptions by
    your own content.\\

    \warn{
      If you are part of a company, the \texttt{Forewords} chapter is
      the best suited place to write about your business, the context
      of this project, introduce yourself and/ou your team, etc.
    }

%******************************************************************************%
%                                                                              %
%                                 Introduction                                 %
%                                                                              %
%******************************************************************************%
\chapter{Introduction}

    Introduction is a presentation of the project outline. It is valued
    to provide some context and some ideas about what needs to be done.
    Thus reading these few lines, a student has access to a global overview.



%******************************************************************************%
%                                                                              %
%                                  Goals                                       %
%                                                                              %
%******************************************************************************%
\chapter{Goals}

    This chapter introduces the pedagogical interests of your project,
    because in the end, a project is only a mean to explore and
    discover new topics. For instance our \texttt{42} \texttt{C++}
    project \texttt{Nibbler}. Despite being just a simple
    \texttt{Snake} game, this project introduces the students to the
    creation of an API and some plugins for a \texttt{C++} program.



%******************************************************************************%
%                                                                              %
%                             General instructions                             %
%                                                                              %
%******************************************************************************%
\chapter{General instructions}

    This chapter lists all basic instructions of a project.
    Language, restrictions, permissions, compilation, etc.



%******************************************************************************%
%                                                                              %
%                             Mandatory part                                   %
%                                                                              %
%******************************************************************************%
\chapter{Mandatory part}

    Heart of the subject, the mandatory part describes in details the
    work expected and the possible tools and/or technologies
    required. The secret of a good subject is the balance between
    being specific and leaving a part to the interpretation and
    imagination. This balance is very important as it is the engine
    that fuels debates and argumentations during peer-evaluation.


%******************************************************************************%
%                                                                              %
%                             Exercises of a Piscine                           %
%                                                                              %
%******************************************************************************%

\chapter{Exercise \exercicenumber: My First Method}

\extitle{Title}
\exnumber{\exercicenumber}
\exscore{2}
\exfiles{my\_first\_method.rb}
\exauthorize{All}

\makeheaderfiles

%******************************************************************************%
%                                                                              %
%                                 Bonus part                                   %
%                                                                              %
%******************************************************************************%
\chapter{Bonus part}

    When a student invests time in a project and the goals are met,
    it's innate to will to go further ! The bonus section is here to satisfy
    such ambition. Of course, the bonus part is exclusively available
    if and only if the mandatory part is complete and perfect.



%******************************************************************************%
%                                                                              %
%                           Turn-in and peer-evaluation                        %
%                                                                              %
%******************************************************************************%
\chapter{Turn-in and peer-evaluation}

    This part describes the conditions and instructions regarding the turn-in and
    the peer-evaluation of the project. If your project does not
    require odd turn-in or peer-evaluation instructions, feel free to
    use the following paragraph as it is:\\

    Turn your work in using your \texttt{GiT} repository, as
    usual. Only work present on your repository will be graded in defense.



%******************************************************************************%
\end{document}
