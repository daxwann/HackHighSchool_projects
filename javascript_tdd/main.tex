%******************************************************************************%
%                                                                              %
%                  javascript_tdd.en.tex for LaTeX                             %
%                  Created on : Tue Mar 10 13:27:28 2015                       %
%                  Made by : Dax Wann                                          %
%                                                                              %
%******************************************************************************%

\documentclass{42-en}

%******************************************************************************%
%                                                                              %
%                                    Header                                    %
%                                                                              %
%******************************************************************************%
\begin{document}



                           \title{Test Driven Development (TDD) in Javascript}
                          \subtitle{Code -> Yay! -> Code -> Yay!}
                       \member{Dax Wann}{daxwann@gmail.com}
                        \member{42 Staff}{pedago@42.fr}

\summary {
  An introduction to unit testing and TDD workflow.
}

\maketitle

\tableofcontents


%******************************************************************************%
%                                                                              %
%                                  Foreword                                    %
%                                                                              %
%******************************************************************************%
\chapter{Foreword}

    \begin{figure}[H]
        \begin{center}
            \includegraphics[width=8cm]{sans_soleil.jpg}
        \end{center}
    \end{figure}
    "The first image he told me about was of three children on a road in Iceland, in 1965. He said that for him it was the image of happiness and also that he had tried several times to link it to other images, but it never worked. He wrote me: one day I'll have to put it all alone at the beginning of a film with a long piece of black leader; if they don't see happiness in the picture, at least they'll see the black. \\
    
    
    He wrote: I'm just back from Hokkaido, the Northern Island. Rich and hurried Japanese take the plane, others take the ferry: waiting, immobility, snatches of sleep. Curiously all of that makes me think of a past or future war: night trains, air raids, fallout shelters, small fragments of war enshrined in everyday life. He liked the fragility of those moments suspended in time. Those memories whose only function had been to leave behind nothing but memories. He wrote: I've been round the world several times and now only banality still interests me. On this trip I've tracked it with the relentlessness of a bounty hunter. At dawn we'll be in Tokyo.\\
    
    He used to write me from Africa. He contrasted African time to European time, and also to Asian time. He said that in the 19th century mankind had come to terms with space, and that the great question of the 20th was the coexistence of different concepts of time. By the way, did you know that there are emus in the Île de France?"\\
    
    - \textit{Sans Soleil}, 1983\\

%******************************************************************************%
%                                                                              %
%                                 Introduction                                 %
%                                                                              %
%******************************************************************************%
\chapter{Introduction}

As coders, when we are facing a programming task, we are inclined to start coding with a solution in mind. It's natural for us to continue coding until we think we have solved the problem. Then we execute this code, hoping it would work the first time. But more often than not, the program crashes or does not do what we intended. And we get frustrated. If the program is small enough, we can take some time to debug to fix the program. However, if the program is huge, the task of debugging will be overwhelming and time consuming.\\
    
Instead of coding and coding until we inevitably encounter an insurmountable error, we will code by repeating these little steps: \\
    1) Before we code one small unit of the program, we create an unit test for it. Watch its test cases fail. RED \\
    2) We write just enough code for this unit in order for the unit test to pass. GREEN \\
    3) We organize the code to make it look cleaner and run more efficiently. Then make sure it passes the same unit test again. REFACTOR \\
    
Red -> Green -> Refactor -> Repeat! the three simple steps of Test Driven Development. 

%******************************************************************************%
%                                                                              %
%                                  Goals                                       %
%                                                                              %
%******************************************************************************%
\chapter{Goals}

    This chapter introduces the pedagogical interests of your project,
    because in the end, a project is only a mean to explore and
    discover new topics. For instance our \texttt{42} \texttt{C++}
    project \texttt{Nibbler}. Despite being just a simple
    \texttt{Snake} game, this project introduces the students to the
    creation of an API and some plugins for a \texttt{C++} program.



%******************************************************************************%
%                                                                              %
%                             General instructions                             %
%                                                                              %
%******************************************************************************%
\chapter{General instructions}

    This chapter lists all basic instructions of a project.
    Language, restrictions, permissions, compilation, etc.



%******************************************************************************%
%                                                                              %
%                             Mandatory part                                   %
%                                                                              %
%******************************************************************************%
\chapter{Mandatory part}

    Heart of the subject, the mandatory part describes in details the
    work expected and the possible tools and/or technologies
    required. The secret of a good subject is the balance between
    being specific and leaving a part to the interpretation and
    imagination. This balance is very important as it is the engine
    that fuels debates and argumentations during peer-evaluation.


%******************************************************************************%
%                                                                              %
%                             Exercises of a Piscine                           %
%                                                                              %
%******************************************************************************%

\chapter{Exercise \exercicenumber: My First Method}

\extitle{Title}
\exnumber{\exercicenumber}
\exscore{2}
\exfiles{my\_first\_method.rb}
\exauthorize{All}

\makeheaderfiles

%******************************************************************************%
%                                                                              %
%                                 Bonus part                                   %
%                                                                              %
%******************************************************************************%
\chapter{Bonus part}

    When a student invests time in a project and the goals are met,
    it's innate to will to go further ! The bonus section is here to satisfy
    such ambition. Of course, the bonus part is exclusively available
    if and only if the mandatory part is complete and perfect.



%******************************************************************************%
%                                                                              %
%                           Turn-in and peer-evaluation                        %
%                                                                              %
%******************************************************************************%
\chapter{Turn-in and peer-evaluation}

    This part describes the conditions and instructions regarding the turn-in and
    the peer-evaluation of the project. If your project does not
    require odd turn-in or peer-evaluation instructions, feel free to
    use the following paragraph as it is:\\

    Turn your work in using your \texttt{GiT} repository, as
    usual. Only work present on your repository will be graded in defense.



%******************************************************************************%
\end{document}
